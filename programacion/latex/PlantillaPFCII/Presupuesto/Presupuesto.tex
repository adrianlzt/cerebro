\chapter{PRESUPUESTO DEL PROYECTO}
En este ap�ndice se presentan justificados los costes globales de
la realizaci�n de este Proyecto Fin de Carrera. Tales costes,
imputables a gastos de personal y de material, se pueden deducir
de las Tablas~\ref{tab:presupuesto1} y~\ref{tab:presupuesto2}.

\begin{sf}
    \begin{table}
        \begin{center}
            \caption[Fases del Proyecto]{\it\small \label{tab:presupuesto1} Fases del Proyecto}
            \vspace{2pt}
            \fbox{\begin{tabular}{|c|c|c|}


                \hline
                \emph{\textbf{Fase 1}} & \emph{Documentaci�n} & $350$ \emph{horas} \\
                \hline
                \emph{\textbf{Fase 2}} & \emph{Desarrollo del software} & $90$ \emph{horas} \\
                \hline
                \emph{\textbf{Fase 3}} & \emph{An�lisis de la base de datos} &  $500$ \emph{horas} \\
                \hline
                \emph{\textbf{Fase 4}} & \emph{Redacci�n de la memoria del proyecto} & $250$ \emph{horas} \\
                \hline

            \end{tabular}}
        \end{center}
    \end{table}
\end{sf}

En la Tabla~\ref{tab:presupuesto1} se muestran las fases del
proyecto y el tiempo aproximado para cada una de ellas. As� pues,
se desprende que  el tiempo total dedicado por el proyectando ha
sido de 1.190 horas, de las cuales aproximadamente un 30\% han
sido compartidas con el tutor del proyecto, por lo que el total
asciende a 1.547 horas. Teniendo en cuenta que la tabla de
honorarios del Colegio Oficial de Ingenieros T�cnicos de
Telecomunicaci�n establece unas tarifas de 60 \euro$/$hora, el
coste de personal se sit�a en 92.820 \euro.

\begin{sf}
    \begin{table}
        \begin{center}
            \caption[Costes de material]{\it\small \label{tab:presupuesto2} Costes de material}
            \vspace{2pt}
            \fbox{\begin{tabular}{|c|c|}


                \hline
                \emph{Ordenador de gama media} & 1.300 \euro \\
                \hline
                \emph{Local (durante 12 meses, con un coste de 120 \euro/mes}) & 1.440 \euro \\
                \hline
                \emph{Documentaci�n} & 200 \euro \\
                \hline
                \emph{Gastos varios} & 700 \euro \\
                \hline

            \end{tabular}}
        \end{center}
    \end{table}
\end{sf}

En la Tabla~\ref{tab:presupuesto2} se recogen los costes de
material desglosados en equipo inform�tico, local de trabajo,
documentaci�n y gastos varios no atribuibles (material fungible,
llamadas telef�nicas, desplazamientos...). Ascienden, pues, a un
total de 3.640 \euro.

A partir de estos datos, el presupuesto total es el mostrado en la
Tabla~\ref{tab:presupuesto3}.

\begin{sf}
    \begin{table}
        \begin{center}
            \caption[Presupuesto]{\it\small \label{tab:presupuesto3} Presupuesto}
            \vspace{2pt}
            \fbox{\begin{tabular}{|c||c|}


                \hline
                \textbf{Concepto} & \textbf{Importe} \\
                \hline
                Costes personal & 78.000 \euro \\
                \hline
                Costes material & 3.640 \euro \\
                \hline
                Base imponible & 96.460 \euro\\
                \hline
                I.V.A. ($16\%$) & 15.433,6 \euro \\
                \hline
                TOTAL & 111.893,6 \euro \\
                \hline

            \end{tabular}}
        \end{center}
    \end{table}
\end{sf}
