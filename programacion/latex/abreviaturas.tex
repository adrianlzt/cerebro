% http://en.wikibooks.org/wiki/LaTeX/Glossary
% http://texblog.org/2014/01/15/glossary-and-list-of-acronyms-with-latex/
% http://get-software.net/macros/latex/contrib/glossaries/glossaries-user.pdf
% http://tex.stackexchange.com/questions/100354/list-of-abbreviations/100377#100377


%% glossaries
% es la pagina donde nos dice en que pagina se ha usado cada palabra
\documentclass{article}
\usepackage{glossaries}
\usepackage[toc]{glossaries} % si queremos que aparezca en el indice
\usepackage[acronym]{glossaries} % si queremos generar una lista aparte solo con los acronimos
\usepackage[toc,acronym,nonumberlist,seeautonumberlist]{glossaries} % generar indice, distinta lista para acronimos, no poner numero de pagina
% la ultima opcion no se muy bien para que es, pero parece que es mejor ponerla

\makeglossaries % antes del \begin{document}

% la idea es tener un glosario.tex donde definamos todas palabras y hacer un \include{glosario} desde nuestro tex general
% los documentos solo aparecerán en el \makeglossaries si es usado en el documento
\newglossaryentry{servidor}{
name={servidor}, %obligatorio
plural={servidores},
description={nodo que, formando parte de una red, provee servicios a otros nodos denominados clientes}
} 


\begin{document}
Here is some text, where we use \gls{servidor}.
\glspl{servidor} % pondra servidores. Si no definimos plural pondra una 's' al final
\Gls{servidor} %primera letra en mayuscula, 'Servidor'

\include{glosario}
\printglossary[type=\acronymtype] % prints just the list of acronyms
\renewcommand{\glsnamefont}[1]{\bf{#1}} % palabras del listado en negrita
\setlength{\glsdescwidth}{0.8\linewidth} % ocupar mas ancho del documento. 1.0 descripción enana llevada al limite de la pagina
\printglossary[style=long]
\end{document}
% para generar el glosario:
% pdflatex nombre; makeglossaries nombre; pdflatex nombre
% otra forma?
% pdflatex; makeindex -s liselot2.ist -t liselot2.glg -o liselot2.gls liselot2.glo; pdflatex

% Si tenemos problemas con la compilacion borrar el fichero *.gls



% Para generar enlaces cargar el paquete hyperref antes del glossaries
% No me funciona con ezthesis. Mirar pagina 89 http://osl.ugr.es/CTAN/macros/latex/contrib/glossaries/glossaries-user.pdf



Ejemplos:

\newglossaryentry{css}
{
  type=\acronymtype,
  name={CSS},
  description={Cascading Style Sheets},
  first={Cascading Style Sheets (CSS)},
  see=[Glossary:]{cssg}
}
\newglossaryentry{cssg} {
  name={Cascading Style Sheets},
  description={A language for specifying presentation attributed of XML documents}
}


\newglossaryentry{computer}
{
  name=computer,
  description={is a programmable machine that receives input,
               stores and manipulates data, and provides
               output in a useful format}
}

\newglossaryentry{Linux}
{
  name=Linux,
  description={is a generic term referring to the family of Unix-like
               computer operating systems that use the Linux kernel},
  plural=Linuces
}

\newacronym[longplural=Frames per Second]{fpsLabel}{FPS}{Frame per Second}

\newacronym{lvm}{LVM}{Logical Volume Manager}

