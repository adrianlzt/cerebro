% http://en.wikibooks.org/wiki/LaTeX/Document_Structure

\documentclass{article}
% preambulo

\begin{document}

% cuerpo del documento
\maketitle

\tableofcontents %puede que tengamos que compilar dos o tres veces

\emph{texto} %para resaltar, normalmente en italica

% dividir el documento en secciones
\part{..} % Dividir un documento en partes independientes. Obtiene numeracion romana
\chapter{..}
\section{..}
\subsection{..} %aparece en el indice. Ej.: 8.7.1
\subsubsection{..} %no aparece en el indice. Solo pone la palabra en negrita y mete un cambio de linea
\paragraph{..} %pone la palabra en negrita. No mete cambio de linea, solo una separación un poco más grande con lo que se escriba a continuacion
\subparagraph{..} %como el paragraph pero tabulado más a la derecha

Si queremos que en el TOC aparezca algo distinto:
\section[Effect on staff turnover]{An analysis of the effect of the revised recruitment policies on staff turnover at divisional headquarters}



% aqui generalmente se hacen includes (\include{nombre}) de otroso ficheros latex

\end{document}
