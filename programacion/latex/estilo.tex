% negrita
% toda la linea. Puede que afecte a siguientes lineas?
\bfseries Esta es una linea en negrita

% Solo una/s palabra/s
{\bfseries Titulo:} La obra


\textit{en cursiva}
\textsc{En Mayúsculas}
\emph{enfatizado} % \emph{palabra\_con\_guiones\_bajos}
\underline{subrayado}
\texttt{como máquina de escribir}
\textsl{inclinada (no cursiva)}...


Si \emph{} rompe el alineamiento (no respeta los márgenes) podemos poner en el parrafo:
\begin{sloppypar}
blabla blab ablablalba \emph{blab aasd a dasd ad asd asda}
\end{sloppypar}




% Que el siguiente elemento tenga una determinada separacion vertical
\vspace{2cm}

% Que el siguiente elemento tenga una determinada separacion horizontal
% No siempre me funciona
\hspace{2cm}

% Alineacion de texto
\raggedright
\raggedleft
\centering

% Identar o no identar un parrafo
\indent
\noindent

% rellenar hasta el final de pagina
\vfill

% lineas con doble espaciado (mas distancia entre lineas)
\begin{doublespace}
  Fecha de lectura:\\
  Califaci'on:
\end{doublespace}

% pagina en blanco
\newpage
\thispagestyle{empty}
\mbox{}
% el estilo debe ser uno de:
%  plain - Just a plain page number.
%  empty - Produces empty heads and feet - no page numbers.
%  headings - Puts running headings on each page. The document style specifies what goes in the headings.
%  myheadings - You specify what is to go in the heading with the \markboth or the \markright commands.

% para documentos doublesided (creo que solo vale para estos)
% es para poner la pagina de la izquierda en blanco, si es la de la derecha, no la dejara en blanco
\usepackage{emptypage}
\cleardoublepage
